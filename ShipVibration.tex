\documentclass{beamer}

\usepackage{multicol}

%-------Metropolis Theme----------------------------------
\usetheme[progressbar=frametitle]{metropolis}
\setbeamertemplate{frame numbering}[fraction]
\useoutertheme{metropolis}
\useinnertheme{metropolis}
\usecolortheme{spruce}
\setbeamercolor{background canvas}{bg=white}
\setbeamercovered{transparent=3} 

\definecolor{mygreen}{rgb}{.125,.5,.25}
\usecolortheme[named=mygreen]{structure}

%----------------------Custom color-----------------------
%\definecolor{celadon}{rgb}{0.67,0.88,0.69}
%\usetheme[progressbar=frametitle]{metropolis}
%\useoutertheme{metropolis}
%\useinnertheme{metropolis}
%\setbeamercolor{background canvas}{bg=white}
%
%\setbeamercolor{frametitle}{bg=celadon}
%\setbeamercolor{progress bar}{fg=black}
%---------------------------------------------------------

\title{Ship Vibration Problems and Remedies}
\author{Eshrak Kader}
%\subtitle{}
%\institute{}
%\institute{\large \textbf{Institute goes here}: \\[6pt]Text here}
\date{\today}


\begin{document}
\metroset{block=fill} %for gray block around a sentence

\begin{frame}
\titlepage
\end{frame}

\begin{frame}[t]{Table of Contents}\vspace{10pt}
\begin{enumerate}
\item Introduction
\item Vibration Sources
\item Vibration Identification
\item Vibration Mitigation
\item Conclusion
\item Bibliography
\end{enumerate}
\end{frame}

\section{Introduction}
	\begin{frame}[t]{Introduction}\vspace{10pt}
	Ship vibration is attributed to unbalanced 						reciprocating machinery, hull vibration due to wave action, 		propeller excitation etc.\vspace{10pt}

	In the present day, ship vibration is of such importance 			that it is taken into account in the design stage. The crux 		of the matter is to ensure that natural frequency of the 			hull and machinery does not match within the ship service 			speed range so as to give rise to resonant condition. This 			requires elaborate calculations of all the significant modes 		of vibration.
	\end{frame}

\section{Vibration Sources}
	\begin{frame}[t]{Vibration Sources}\vspace{10pt}
	\begin{block}{The potential sources of vibration in ships 			are:}
	\end{block}
	\begin{enumerate}
	\item Local vibration like rattling, creaking in fittings and 	engine vibration.
	\item Rudder vibration caused by torsional oscillation and 			bending oscillation, fluttering of rudder.
	\item Propeller excitation causing axial and torsional 				vibration of propeller shaft.
	\item Engine excitation due to external unbalance and 				flywheel effect.
	\end{enumerate}
	\end{frame}

\section{Vibration Identification}
	\begin{frame}[t]{Vibration Identification}\vspace{10pt}
	\begin{block}{Key observations to identify vibration:}
	\end{block}
	\begin{enumerate}
	\item Engine rpm corresponding to significant vibration.
	\item Direction of vibration, transverse or longitudinal.
	\item Frequency of vibration.
	\item Position of maximum and minimum vibration.
	\item Nature of local vibration.
	\item Engine rpm causing hammering of steering gear and tail 		shaft.
	\end{enumerate}
	\end{frame}

\section{Vibration Mitigation}
	\begin{frame}[t]{Vibration Mitigation}\vspace{10pt}
	\begin{block}{The remedy for ship vibration include:}
	\end{block}
	\begin{enumerate}
	\item Installing vibration insulating materials, firmly 			securing bolts and screws, using engine mounting made of 			resilient materials and increasing scantlings in areas prone 		to vibration.
	\item Rudder design prioritizing natural frequency over max 		operating speed. To avoid flutter rudder must be designed to 		have maximum rigidity against torsion and flexure.
	\item Propeller design ensuring smooth water flow, adequate 		clearance between blade tip and hull and sufficient 				structural stiffness of propeller disc.
	\item Favorable firing order of the engine and fitting 				crankweb balance to negate resultant moment.
	\end{enumerate}
	\end{frame}

\section{Conclusion}
	\begin{frame}[t]{\textbf{Conclusion}}\vspace{10pt}
	Vibration has plagued the shipping industry since its 				inception, with the advent of new propulsion systems the 			source of vibration may have altered, however the challenge 		of mitigating vibration in order to ensure sound ship 				operation and the well being
	of her crew and passenger is a top priority.\vspace{5pt}

	Decades of research and practical know how has led to some 			innovative solutions to this ubiquitous problem and the era 		of electronic computing has granted the opportunity to model 		ship vibration such that the present day ships are more 			optimized in design to counter phenomenon and are oriented 			towards crew and passenger comfort.
	\end{frame}

\section{Bibliography}
	\begin{frame}[t]{\textbf{Bibliography}}\vspace{10pt}
	\begin{enumerate}
	\item "A review of ship vibration problem", W. Ker Wilson, 			1956.
	\item "Practical Approach to Some Vibration and Machinery 			Problems", T. W. Bunyan, 1955.
	\end{enumerate}
	\end{frame}


\begin{frame}[standout]
\begin{center}
Thank You.
\end{center}
\end{frame}

\end{document}